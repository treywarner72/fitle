\subsection{Example: Convolutional Fit}
The key method in this work was to convolve QED-based radiative predictions (\texttt{PHOTOS} simulations) with the approximate resolution function.
\begin{equation}
\label{eq:convolution_approx_2}
(Q \star R)(x) \;\approx\;
\sum_{n=0}^{n_{\mathrm{bins}} - 1}
Q(x'_n)\, R\!\bigl(x - x'_n\bigr)\,\Delta x'_n
\end{equation}
where $Q$ denotes the radiative distribution and $R$ the resolution function (e.g.\ double Gaussian).
In \texttt{fitle}, discrete convolution can be implemented (efficiently) with indices and reductions:
\begin{lstlisting}[language=Python]
    from fitle import INPUT, INDEX, Param, gaussian, const, Reduction, index
    import operator

    def convolve(centers, counts, mass_mother, mu, sigma, bin_width=1.0):
        n = index(len(counts))
        Qx_n = const(counts)[n]
        x_n = const(centers)[n]
        shifted_x = INPUT + mass_mother - mu
        R = gaussian(mu=x_n, sigma=sigma) % shifted_x
        weighted = Qx_n * R
        ret = Reduction(weighted, n, operator.add)
        return ret / (bin_width * np.sum(ret))
\end{lstlisting}
In this function, \texttt{index} creates \texttt{n} which is an index parameter, a kind of \texttt{Param} that stores ranges.
The indexing syntax \texttt{const(counts)[n]} creates a \texttt{Model} that represents the \texttt{n}th count of a histogram (such as a PHOTOS histogram); similarly \texttt{const(centers)[n]} selects the \texttt{n}th center.
The \texttt{\%} operator replaces elements in a model with a map, but in the absence of a dictionary, as in the code above, we replace \texttt{INPUT} by default.
The $\Delta x'_n$ term of Eq.~\ref{eq:convolution_approx_2} is not included explicitly as it is already included in the \texttt{counts} of the PHOTOS histogram.
The \texttt{Reduction} of \texttt{weighted} over \texttt{n} using addition represents the summation in Eq.~\ref{eq:convolution_approx_2}.
Finally, we normalize by dividing by the bin width and the sum of the result.

This convolution is available as a built-in function:
\begin{lstlisting}[language=Python]
    from fitle import Param, convolve, fit, Cost

    mu = Param('mu')(1968)
    sigma = (+Param)(5)

    model = +Param * convolve(
        d_x=centers, c=counts, mass_mother=1968,
        mu=mu, sigma=sigma, bin_width=1.0
    )

    fit_result = fit(model | Cost.chi2(data, 200))
\end{lstlisting}
